\documentclass[12pt, psamsfonts]{amsart}

%-------Packages---------
\usepackage{amssymb,amsfonts}
\usepackage{semantic}
\usepackage{fullpage}
\usepackage{tikz-cd}
\usepackage{todonotes}
\usepackage{physics}
\usepackage[all,arc]{xy}
\usepackage{enumerate}
\usepackage{enumitem}
\usepackage{mathrsfs}
\usepackage{theoremref}
\usepackage{graphicx}
\usepackage[bookmarks]{hyperref}

%--------Theorem Environments--------
%theoremstyle{plain} --- default
\newtheorem{thm}{Theorem}[section]
\newtheorem{cor}[thm]{Corollary}
\newtheorem{prop}[thm]{Proposition}
\newtheorem{lem}[thm]{Lemma}
\newtheorem{conj}[thm]{Conjecture}
\newtheorem{quest}[thm]{Question}

\theoremstyle{definition}
\newtheorem{defn}[thm]{Definition}
\newtheorem{defns}[thm]{Definitions}
\newtheorem{con}[thm]{Construction}
\newtheorem{exmp}[thm]{Example}
\newtheorem{exmps}[thm]{Examples}
\newtheorem{notn}[thm]{Notation}
\newtheorem{notns}[thm]{Notations}
\newtheorem{addm}[thm]{Addendum}
\newtheorem*{exer}{Exercise}

\theoremstyle{remark}
\newtheorem{rem}[thm]{Remark}
\newtheorem{rems}[thm]{Remarks}
\newtheorem{warn}[thm]{Warning}
\newtheorem{sch}[thm]{Scholium}

\DeclareMathOperator{\Hom}{Hom}
\DeclareMathOperator{\Id}{Id}
\DeclareMathOperator{\End}{End}
\DeclareMathOperator{\ord}{ord}
\DeclareMathOperator{\Aut}{Aut}
\DeclareMathOperator{\Gal}{Gal}

\makeatletter
\let\c@equation\c@thm
\makeatother
\numberwithin{equation}{section}

\bibliographystyle{plain}

\begin{document}

\title{Qualifying Exam Prep}
\author{Hidenori Shinohara}
\maketitle

\begin{abstract}
  In order to prepare for the qualifying exam, I decided to solve problems from Hatcher and Dummit and Foote.
\end{abstract}

\tableofcontents

\section{Algebra}

\subsection{Groups}
The topics to cover: Elementary concepts (homomorphism, subgroup, coset, normal subgroup), solvable groups, commutator subgroup, Sylow theorems, structure of finitely generated Abelian groups.
Symmetric, alternating, dihedral, and general linear groups.

\subsection{Rings}
The topics to cover: Commutative rings and ideals (principal, prime, maximal).
Integral domains, Euclidean domains, principal ideal domains, polynomial rings, Eisenstein's irreducibility criterion, Chinese remainder theorem.
Structure of finitely generated modules over a principal ideal domain.

\subsubsection{Chinese remainder theorem}

\begin{exer}{(Problem 1, Section 7.6)}
  Let $R$ be a ring with identity $1 \ne 0$.
  An element $e \in R$ is called an idempotent if $e^2 = e$.
  Assume $e$ is an idempotent in $R$ and $er = re$ for all $r \in R$.
  Prove that $Re$ and $R(1 - e)$ are two-sided ideals of $R$ and that $R \cong Re \times R(1 - e)$.
  Show that $e$ and $1 - e$ are identities for the subrings $Re$ and $R(1 - e)$ respectively.
\end{exer}

\begin{proof}
  $Re$ is clearly nonempty and $re + r'e = (r + r')e \in Re$ for all $re, r'e \in Re$.
  For all $r' \in R$ and $re \in Re$, $r'(re) = (r'r)e \in Re$ and $(re)r' = r(er') = r(r'e) = (rr')e \in Re$.
  Thus $Re$ is a two-sided ideal of $R$.
  $(1 - e)^2 = 1 - e - e + e^2 = 1 - e$, and, for every $r \in R$, $r(1 - e) = r - re = r - er = (1 - e)r$.
  Thus $R(1 - e)$ is a two-sided ideal of $R$.
  Finally, $\phi: R \rightarrow R / Re \times R / R(1 - e)$ defined by $x \mapsto (x + Re, x + R(1 - e))$ is a ring homomorphism with $\ker(\phi) = Re \cap R(1 - e)$ by the Chinese Remainder Theorem.
  Let $r(1 - e) \in \ker(\phi) = Re \cap R(1 - e)$.
  Then $r(1 - e)e = r(1 - e)$ since $r(1 - e) \in Re$.
  However, this implies $r(1 - e)e = r(e - e^2) = r0 = 0$.
  Thus $\ker(\phi) = 0$, so $R \cong R / Re \times R / R(1 - e)$.
\end{proof}


\subsection{Fields Extensions}
Finite, algebraic, separable, inseparable, transcendental, splitting field of a polynomial, primitive element theorem, algebraic closure.
Finite fields.

\subsection{Galois Theory}
Finite Galois extensions and the Galois correspondence between subgroups of the Galois group and sub-extensions.
Solvable extensions and solving equations by radicals.

\section{Algebraic topology}

\subsection{Fundamental group}
Computation of the fundamental group, van Kampen's theorem, covering spaces.

\subsection{Homology}
Singular chains, chain complexes, homotopy invariance.
Relationship between the first homology and the fundamental group, relative homology.
The long exact sequence of relative homology.
The Mayer-Vietoris sequence.



\end{document}


